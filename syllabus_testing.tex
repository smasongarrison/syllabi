\documentclass[11pt,]{article}
\usepackage[margin=1in]{geometry}
\newcommand*{\authorfont}{\fontfamily{phv}\selectfont}
\usepackage[]{mathpazo}
\usepackage{abstract}
\renewcommand{\abstractname}{}    % clear the title
\renewcommand{\absnamepos}{empty} % originally center
\newcommand{\blankline}{\quad\pagebreak[2]}

\providecommand{\tightlist}{%
  \setlength{\itemsep}{0pt}\setlength{\parskip}{0pt}}
\usepackage{longtable,booktabs}

\usepackage{parskip}
\usepackage{titlesec}
\titlespacing\section{0pt}{12pt plus 4pt minus 2pt}{6pt plus 2pt minus 2pt}
\titlespacing\subsection{0pt}{12pt plus 4pt minus 2pt}{6pt plus 2pt minus 2pt}

\titleformat*{\subsubsection}{\normalsize\itshape}

\usepackage{titling}
\setlength{\droptitle}{-.25cm}

%\setlength{\parindent}{0pt}
%\setlength{\parskip}{6pt plus 2pt minus 1pt}
%\setlength{\emergencystretch}{3em}  % prevent overfull lines

\usepackage[T1]{fontenc}
\usepackage[utf8]{inputenc}

\usepackage{fancyhdr}
\pagestyle{fancy}
\usepackage{lastpage}
\renewcommand{\headrulewidth}{0.3pt}
\renewcommand{\footrulewidth}{0.0pt}
\lhead{}
\chead{}
\rhead{\footnotesize PSY 362. Psychological Testing -- November 22, 2020}
\lfoot{}
\cfoot{\small \thepage/\pageref*{LastPage}}
\rfoot{}

\fancypagestyle{firststyle}
{
\renewcommand{\headrulewidth}{0pt}%
   \fancyhf{}
   \fancyfoot[C]{\small \thepage/\pageref*{LastPage}}
}

%\def\labelitemi{--}
%\usepackage{enumitem}
%\setitemize[0]{leftmargin=25pt}
%\setenumerate[0]{leftmargin=25pt}




\makeatletter
\@ifpackageloaded{hyperref}{}{%
\ifxetex
  \usepackage[setpagesize=false, % page size defined by xetex
              unicode=false, % unicode breaks when used with xetex
              xetex]{hyperref}
\else
  \usepackage[unicode=true]{hyperref}
\fi
}
\@ifpackageloaded{color}{
    \PassOptionsToPackage{usenames,dvipsnames}{color}
}{%
    \usepackage[usenames,dvipsnames]{color}
}
\makeatother
\hypersetup{breaklinks=true,
            bookmarks=true,
            pdfauthor={ ()},
             pdfkeywords = {},
            pdftitle={PSY 362. Psychological Testing},
            colorlinks=true,
            citecolor=blue,
            urlcolor=blue,
            linkcolor=magenta,
            pdfborder={0 0 0}}
\urlstyle{same}  % don't use monospace font for urls


\setcounter{secnumdepth}{0}

\usepackage{color}
\usepackage{fancyvrb}
\newcommand{\VerbBar}{|}
\newcommand{\VERB}{\Verb[commandchars=\\\{\}]}
\DefineVerbatimEnvironment{Highlighting}{Verbatim}{commandchars=\\\{\}}
% Add ',fontsize=\small' for more characters per line
\usepackage{framed}
\definecolor{shadecolor}{RGB}{248,248,248}
\newenvironment{Shaded}{\begin{snugshade}}{\end{snugshade}}
\newcommand{\KeywordTok}[1]{\textcolor[rgb]{0.13,0.29,0.53}{\textbf{#1}}}
\newcommand{\DataTypeTok}[1]{\textcolor[rgb]{0.13,0.29,0.53}{#1}}
\newcommand{\DecValTok}[1]{\textcolor[rgb]{0.00,0.00,0.81}{#1}}
\newcommand{\BaseNTok}[1]{\textcolor[rgb]{0.00,0.00,0.81}{#1}}
\newcommand{\FloatTok}[1]{\textcolor[rgb]{0.00,0.00,0.81}{#1}}
\newcommand{\ConstantTok}[1]{\textcolor[rgb]{0.00,0.00,0.00}{#1}}
\newcommand{\CharTok}[1]{\textcolor[rgb]{0.31,0.60,0.02}{#1}}
\newcommand{\SpecialCharTok}[1]{\textcolor[rgb]{0.00,0.00,0.00}{#1}}
\newcommand{\StringTok}[1]{\textcolor[rgb]{0.31,0.60,0.02}{#1}}
\newcommand{\VerbatimStringTok}[1]{\textcolor[rgb]{0.31,0.60,0.02}{#1}}
\newcommand{\SpecialStringTok}[1]{\textcolor[rgb]{0.31,0.60,0.02}{#1}}
\newcommand{\ImportTok}[1]{#1}
\newcommand{\CommentTok}[1]{\textcolor[rgb]{0.56,0.35,0.01}{\textit{#1}}}
\newcommand{\DocumentationTok}[1]{\textcolor[rgb]{0.56,0.35,0.01}{\textbf{\textit{#1}}}}
\newcommand{\AnnotationTok}[1]{\textcolor[rgb]{0.56,0.35,0.01}{\textbf{\textit{#1}}}}
\newcommand{\CommentVarTok}[1]{\textcolor[rgb]{0.56,0.35,0.01}{\textbf{\textit{#1}}}}
\newcommand{\OtherTok}[1]{\textcolor[rgb]{0.56,0.35,0.01}{#1}}
\newcommand{\FunctionTok}[1]{\textcolor[rgb]{0.00,0.00,0.00}{#1}}
\newcommand{\VariableTok}[1]{\textcolor[rgb]{0.00,0.00,0.00}{#1}}
\newcommand{\ControlFlowTok}[1]{\textcolor[rgb]{0.13,0.29,0.53}{\textbf{#1}}}
\newcommand{\OperatorTok}[1]{\textcolor[rgb]{0.81,0.36,0.00}{\textbf{#1}}}
\newcommand{\BuiltInTok}[1]{#1}
\newcommand{\ExtensionTok}[1]{#1}
\newcommand{\PreprocessorTok}[1]{\textcolor[rgb]{0.56,0.35,0.01}{\textit{#1}}}
\newcommand{\AttributeTok}[1]{\textcolor[rgb]{0.77,0.63,0.00}{#1}}
\newcommand{\RegionMarkerTok}[1]{#1}
\newcommand{\InformationTok}[1]{\textcolor[rgb]{0.56,0.35,0.01}{\textbf{\textit{#1}}}}
\newcommand{\WarningTok}[1]{\textcolor[rgb]{0.56,0.35,0.01}{\textbf{\textit{#1}}}}
\newcommand{\AlertTok}[1]{\textcolor[rgb]{0.94,0.16,0.16}{#1}}
\newcommand{\ErrorTok}[1]{\textcolor[rgb]{0.64,0.00,0.00}{\textbf{#1}}}
\newcommand{\NormalTok}[1]{#1}
\usepackage{longtable}




\usepackage{setspace}

\title{PSY 362. Psychological Testing}
\author{S. Mason Garrison}

\date{November 22, 2020}


\begin{document}

		\maketitle
	

		\thispagestyle{firststyle}

%	\thispagestyle{empty}


\noindent \begin{tabular*}{\textwidth}{ @{\extracolsep{\fill}} lr @{\extracolsep{\fill}}}
\hline\\
    &  Class Room: Canvas
%
\\\hline\\
%
 Web: \href{https://wakeforest.instructure.com/courses/17546}{\tt wakeforest.instructure.com/courses/17546}&  \\ %wraps \url{} around any url
 %wraps \url{} around any url
 %wraps \url{} around any url
&  \\
Professor: S. Mason Garrison \\

E-mail: \texttt{\href{mailto:GarrisSM@wfu.edu}{\nolinkurl{GarrisSM@wfu.edu}}}  \\

WFU Office: GREENE 438   \\
Virtual Office: Zoom
\href{https://wakeforest-university.zoom.us/my/smasongarrison}{\tiny\tt wakeforest-university.zoom.us/my/smasongarrison}   \\
Hours: By Appointment
\href{https://calendly.com/smasongarrison/}{\small\tt calendly.com/smasongarrison}  \\


	&  \\
	\hline
\end{tabular*}

\vspace{2mm}

\section{Course Description}\label{course-description}

This course provides an overview of the development and nature of
psychological tests with applications to school counseling, business,
and clinical practice. The purpose of the course is to provide students
with an understanding of the principles of measurement as applied to
group standardized measures of achievement, special aptitude,
intelligence, personality, interests and distress for use in counseling.
Format will consist primarily of lectures along with group participation
activities.

To do well in the course, you should read the assigned material before
class and re-read previously assigned material as the course progresses.
By reading the text before class you will be better prepared to ask
questions and integrate the content of lectures with what was presented
in the text. For synchronous classes, be sure to attend all lectures and
arrive on time. For asynchronous classes be sure to watch all the video
lectures and don't leave them until the last minute. Each topic builds
directly on the previous one. Thus, if you miss one lecture, you run the
risk of being completely lost in the next lecture.

In addition, many professors have implicit (\emph{i.e.}, unspoken)
expectations for college classes. I'm going to explicitly state some of
those unspoken expectations. I suspect that your other professors have
similar expectations -- so this information will help you in your other
classes.

\begin{itemize}
\tightlist
\item
  Read the syllabus.
\item
  Read all the class announcements.
\item
  Read the FAQ and post your questions about the class there.
\item
  Read and follow the instructions for each assignment.
\item
  Read your professor's comments on your submissions -- especially if
  you did not get full credit on that submission.
\item
  Treat canvas messages like email.
\item
  Read and respond to canvas messages from your professor.
\item
  Before you send email your professor, try to answer the question
  yourself by looking at the syllabus, reading the course FAQ, and
  reading the assignment instructions.
\item
  Know when assignment deadlines are.
\item
  If you need additional time or flexibility on an assignment, you
  communicate with your professor before the deadline passes. \#\#
  Course Objectives:
\item
  To acquaint you with the fundamental vocabulary and logic of
  psychological measurement and behavioral assessment.
\item
  To develop your capacity for critical judgment of the adequacy of
  measures purported to assess behavior in the role of theory
  development.
\item
  To acquaint you with some of the relevant literature in personality
  assessment, psychometric theory and practice, and methods of observing
  and measuring behavior.
\item
  To instill in you an appreciation of and an interest in the principles
  and methods of psychometric theory in general and behavior assessment
  in particular.
\item
  This course is not designed to make you into an accomplished
  psychometist (one who gives tests) nor is it designed to make you a
  skilled psychometrician (one who constructs tests), nor will it give
  you ``hands on'' experience with psychometric computer programs.
  Rather it is aimed to allow you to understand the fundamental
  theoretical issues concerning both the psychometrist and the
  psychometrician.
\item
  Because modern psychometrics and statistics may be done using open
  source software such as R, examples will be presented in R.
  Instructions for installing and using R for psychometrics are
  available here.
\end{itemize}

\section{Materials}\label{materials}

\subsection{Texts}\label{texts}

\subsubsection{Required}\label{required}

{[}1{]} A. Anastasi and A. Urbina. \emph{Psychological testing}. Upper
Saddle River, New Jersey: Prentice-Hall, 1997. ISBN: 0-02-303020-8.

\paragraph{How to use the required
text:}\label{how-to-use-the-required-text}

The text (Anastasi and Urbina 1997) is intended to supplement the
lectures. The lectures don't follow the order of chapters in the text
and the text covers some material that won't be covered in the lectures,
i.e., not all the information in each chapter is perfectly pertinent to
the course requirements.

\section{Course Assignments}\label{course-assignments}

\subsection{Tests}\label{tests}

There will be X tests in this course. The best 2 out of 3 exam grades
will be used to determine your total exam grade. You can drop any of the
exams for any reason BUT YOU MUST PASS THE LAST EXAM (held at TBD on Dec
TBD). If you do not pass the final exam, your score on that exam will
count as one of your 2 grades. If you do pass the final exam, then the
best 2 grades will be used (irrespective of your score on the final
exam).

\subsubsection{Test Dates}\label{test-dates}

\begin{itemize}
\item
  Test 1: Friday, 09/25
\item
  Test 2: Friday, 10/30
\item
  Test 3: During Finals Week
\end{itemize}

\subsection{Short Assignments}\label{short-assignments}

There will be three (3) short assignments in this course. These
assignments will allow you to incorporate some of your own interests
into the course. Such interests could be related to your career, another
class you're taking, a hobby you're exploring, or some other random
fancy. They will typically be approximately two (2) pages. I will
provide more specific written guidelines at least two weeks before each
assignment is due.

The best two (2) out of three (3) short assignments will be used to
determine your total short assignments. You can skip one of the
assignments for any reason. \#\#\# Short Assignment Dates

\begin{itemize}
\item
  Assignment 1: Friday, 09/18: Critique of an online personality test.
\item
  Assignment 2: Friday, 10/16: Critique of a personality-related media
  article.
\item
  Assignment 3: Friday, 11/13: Critique of a personality-related YouTube
  video.
\end{itemize}

\subsection{Engagement Activities}\label{engagement-activities}

There will be multiple engagement activities in this course. These
activities will allow you engage with the material for each module.
Details about the specific activities will be provided on canvas.
Students must complete two activities per module. For students in the
blended section, they can attend their weekly in-person session and
count it as an engagement activity.

\subsection{Grading Policy}\label{grading-policy}

Typically, an A- is defined as 90\% of the highest point total in the
class, B- as 80\% of that total, C- as 70 and D- as 60\%. I may shift
these values down to provide a better fit to the actual point
distribution. By scaling to a percentage of the highest point total in
the class, each student has a much better chance of receiving higher
grades than if no re-scaling were done. This curve can only help your
grade.

The full table is provided below:

\begin{Shaded}
\begin{Highlighting}[]
\NormalTok{curve_df=}\KeywordTok{data.frame}\NormalTok{(}\DataTypeTok{Letter=}\KeywordTok{c}\NormalTok{(}\StringTok{"A"}\NormalTok{,}\StringTok{"A-"}\NormalTok{,}\StringTok{"B+"}\NormalTok{,}\StringTok{"B"}\NormalTok{,}\StringTok{"B-"}\NormalTok{,}\StringTok{"C+"}\NormalTok{,}\StringTok{"C"}\NormalTok{,}\StringTok{"C-"}\NormalTok{,}\StringTok{"D+"}\NormalTok{,}\StringTok{"D"}\NormalTok{,}\StringTok{"D-"}\NormalTok{),}\DataTypeTok{Cutoff=}\KeywordTok{c}\NormalTok{(}\FloatTok{0.95}\NormalTok{,}\FloatTok{0.90}\NormalTok{,}\FloatTok{0.87}\NormalTok{,}\FloatTok{0.82}\NormalTok{,}\FloatTok{0.80}\NormalTok{,}\FloatTok{0.77}\NormalTok{,}\FloatTok{0.72}\NormalTok{,}\FloatTok{0.70}\NormalTok{,}\FloatTok{0.67}\NormalTok{,}\FloatTok{0.62}\NormalTok{,}\FloatTok{0.60}\NormalTok{))}

\NormalTok{knitr}\OperatorTok{::}\KeywordTok{kable}\NormalTok{(}
\NormalTok{  curve_df, }\DataTypeTok{caption =} \StringTok{'Full Table'}
\NormalTok{)}
\end{Highlighting}
\end{Shaded}

\begin{longtable}[]{@{}lr@{}}
\caption{Full Table}\tabularnewline
\toprule
Letter & Cutoff\tabularnewline
\midrule
\endfirsthead
\toprule
Letter & Cutoff\tabularnewline
\midrule
\endhead
A & 0.95\tabularnewline
A- & 0.90\tabularnewline
B+ & 0.87\tabularnewline
B & 0.82\tabularnewline
B- & 0.80\tabularnewline
C+ & 0.77\tabularnewline
C & 0.72\tabularnewline
C- & 0.70\tabularnewline
D+ & 0.67\tabularnewline
D & 0.62\tabularnewline
D- & 0.60\tabularnewline
\bottomrule
\end{longtable}

\section{Course Policies}\label{course-policies}

\subsection{Grading Policy}\label{grading-policy-1}

Typically an A- is defined as 90\% of the highest point total in the
class, B- as 80\% of that total, C- as 70 and D- as 60\%. I may shift
these values down to provide a better fit to the actual point
distribution. By scaling to a \% of the highest point total in the
class, each student has a much better chance of receiving higher grades
than if no re-scaling were done.

\begin{itemize}
\tightlist
\item
  200 points of your grade will be determined by two (2) tests.
\item
  100 points of your grade will be determined by two (2) short
  assignments.
\item
  50 points of your grade will be determined by your attendance and
  participation in class. Generally, ask questions and answer them.
\end{itemize}

\subsubsection{Test Dates}\label{test-dates-1}

\begin{itemize}
\item
  Test 1: Thursday, 10/13
\item
  Test 2: During Finals Week, TBD
\end{itemize}

\subsection{Short assignments}\label{short-assignments-1}

There will be two (2) assignments in this course. I will provide more
specific written guidelines at least two weeks before each assignment is
due.

\subsubsection{Short Assignment Dates}\label{short-assignment-dates}

\begin{itemize}
\item
  Assignment 1: Tuesday, 10/06.
\item
  Assignment 2: Thursday, 11/17.
\end{itemize}

\subsection{Class Presence and
Participation.}\label{class-presence-and-participation.}

Class presence and participation points are given to encourage your
active class participation and discussion. You will be rewarded with a
perfect score as long as you frequently come to class and actively
contribute to the class discussion during recitations and lectures.

\subsubsection{Excused absences}\label{excused-absences}

I recognize that occasions arise during the academic year that merit the
excused absence of a student from a scheduled class or laboratory during
which an examination, quiz, or other graded exercise is given. Examples
include participation in sponsored university activities (e.g., debate
team, varsity sports), observance of officially designated religious
holidays, serious personal problems (e.g., serious illness, death of a
member of the student's family), and matters relating to the student's
academic training (e.g., graduate or professional school interviews).
Conflicts arising from personal travel plans or social obligations do
not qualify as excused absences.

Except in the case of true emergencies, a possible excused absence
should be discussed with me as far in advance as possible - you should
not assume that an excused absence will automatically be granted. This
discussion should occur via email. If you also discuss your situation
verbally, please send a summary of the discussion via email to me. An
unexcused absence will result in a zero for any graded work that should
have been performed for or during the missed class.

\subsection{Academic Dishonesty
Policy}\label{academic-dishonesty-policy}

All work submitted for credit must be the student's own and is subject
to the provisions of the Wake Forest Honor Code. Details can be found at
the Student Conduct web site:
\href{https://studentconduct.wfu.edu/honor-system-wfu/}{studentconduct.
wfu.edu/honor-system-wfu}.

\subsection{Accommodations Policy}\label{accommodations-policy}

If you are (or become) learning, sensory, or physically disabled, and
need special course accommodations in class, reading, or any other work
in this course, please contact me to discuss your specific needs as soon
as possible. Students who need reasonable accommodations for
disabilities also should contact the Learning Assistance Center \&
Disability Services \href{https://lac.wfu.edu/}{lac.wfu.edu}.

\section{Classroom Climate}\label{classroom-climate}

I aim to create a learning environment for my students that supports a
diversity of thoughts, perspectives, and experiences, and honors your
identities (including race, gender, class, sexuality, religion, ability,
political affiliation, etc.) To help accomplish this:

\begin{itemize}
\item
  If you have a name and/or set of pronouns that differ from those that
  appear in your official records, please let me know!
\item
  If you feel like your performance in the class is being impacted by
  your experiences outside of class, please don't hesitate to come and
  talk with me. I want to be a resource for you. Remember that you can
  also submit anonymous feedback (which will lead to me making a general
  announcement to the class, if necessary to address your concerns).
\item
  I (like many people) am still in the process of learning about diverse
  perspectives and identities. If something was said in class (by
  anyone) that made you feel uncomfortable, please talk to me about it.
  (Again, anonymous feedback is always an option).
\end{itemize}

\subsection{Department Statement}\label{department-statement}

The Psychology Department values, respects, and celebrates the
experiences, beliefs, and practices stemming from varied cultures and
circumstances (emphasizing, but not limited to, those from historically
underrepresented groups), and our deep commitment to diversity, equity,
and inclusion plays out through coursework, programming by majors, and
research.

\section{Tentative Class Schedule}\label{tentative-class-schedule}

This syllabus is intended to give the student guidance in what may be
covered during the semester and will be followed as closely as possible.
However, I reserve the right to modify, supplement and make changes as
the course needs arise.

\subsection{Week 01, 08/24 - 08/28 :
Introduction}\label{week-01-0824---0828-introduction}

\begin{itemize}
\tightlist
\item
  Tuesday: Chapter 1
\item
  Thursday: Chapter 2
\end{itemize}

\subsection{Week 02, 08/31 - 09/04 : Historical, Ethical, and Social
Considerations in
Testing}\label{week-02-0831---0904-historical-ethical-and-social-considerations-in-testing}

\begin{itemize}
\tightlist
\item
  Tuesday: Chapter 18
\item
  Thursday: Chapter 18
\end{itemize}

\subsection{Week 03, 09/07 - 09/11 : Norms and the Meaning of Test
Scores}\label{week-03-0907---0911-norms-and-the-meaning-of-test-scores}

\begin{itemize}
\tightlist
\item
  Tuesday: Chapter 3 ``Last day to add full-term class''
\item
  Thursday: Chapter 3
\end{itemize}

\subsection{Week 04, 09/14 - 09/18 :
Reliability}\label{week-04-0914---0918-reliability}

\begin{itemize}
\tightlist
\item
  Tuesday: Chapter 4
\item
  Thursday: Chapter 4
\end{itemize}

\subsection{Week 05, 09/21 - 09/25 :
Validity}\label{week-05-0921---0925-validity}

\begin{itemize}
\tightlist
\item
  Tuesday: Chapter 5 + 6
\item
  Thursday: Chapter 5 + 6
\end{itemize}

\subsection{Week 06, 09/28 - 10/02 : More
Validity}\label{week-06-0928---1002-more-validity}

\begin{itemize}
\tightlist
\item
  Tuesday: Chapter 6 ``Last day to drop full-term class''
\item
  Thursday: Chapter 9
\end{itemize}

\subsection{Week 07, 10/05 - 10/09 : Item
Analysis}\label{week-07-1005---1009-item-analysis}

\begin{itemize}
\tightlist
\item
  Tuesday: Chapter 7 (Assignment 1 Due)
\item
  Thursday: Chapter 7
\end{itemize}

\subsection{Week 08, 10/12 - 10/16 : Test
1}\label{week-08-1012---1016-test-1}

\begin{itemize}
\tightlist
\item
  Tuesday: Review
\item
  Thursday: Exam 1
\end{itemize}

\subsection{Week 09, 10/19 - 10/23 : Spring
Break}\label{week-09-1019---1023-spring-break}

\begin{itemize}
\tightlist
\item
  Tuesday: No class
\item
  Thursday: No class
\end{itemize}

\subsection{Week 10, 10/26 - 10/30 : Individual
Differences}\label{week-10-1026---1030-individual-differences}

\begin{itemize}
\tightlist
\item
  Tuesday: Selected readings
\item
  Thursday: Chapter 10
\end{itemize}

\subsection{Week 11, 11/02 - 11/06 : Ability
Testing}\label{week-11-1102---1106-ability-testing}

\begin{itemize}
\tightlist
\item
  Monday: Last day to drop with a grade of ``W''
\item
  Tuesday: Chapter 11
\item
  Thursday: Chapter 12
\end{itemize}

\subsection{Week 12, 11/09 - 11/13 : Personality
Testing}\label{week-12-1109---1113-personality-testing}

\begin{itemize}
\tightlist
\item
  Tuesday: Chapter 13
\item
  Thursday: Chapter 13
\end{itemize}

\subsection{Week 13, 11/16 - 11/20 : Measuring Interests and
Attitudes}\label{week-13-1116---1120-measuring-interests-and-attitudes}

\begin{itemize}
\tightlist
\item
  Tuesday: Chapter 14
\item
  Thursday: Chapter 14 (Assignment 2 Due)
\end{itemize}

\subsection{Week 14, 11/23 - 11/27 : Other Assessment
Techniques}\label{week-14-1123---1127-other-assessment-techniques}

\begin{itemize}
\tightlist
\item
  Tuesday: Chapter 15
\item
  Thursday: Chapter 16
\end{itemize}

\subsection{Week 15, 11/30 - 12/04 : Modern Development and Applications
of
Testing}\label{week-15-1130---1204-modern-development-and-applications-of-testing}

\begin{itemize}
\tightlist
\item
  Tuesday: Supplemental reading
\item
  Thursday: Supplemental reading
\end{itemize}

\subsection{Week 16, 12/07 - 12/11 :
Review}\label{week-16-1207---1211-review}

\begin{itemize}
\tightlist
\item
  Tuesday: Review
\end{itemize}

\subsection{Final Exam}\label{final-exam}

\begin{itemize}
\tightlist
\item
  TBD
\end{itemize}

\section*{References}\label{references}
\addcontentsline{toc}{section}{References}

\hypertarget{refs}{}
\hypertarget{ref-anastasi1997}{}
Anastasi, Anne, and A. Urbina. 1997. \emph{Psychological Testing}. Upper
Saddle River, New Jersey: Prentice-Hall.





\end{document}

\makeatletter
\def\@maketitle{%
  \newpage
%  \null
%  \vskip 2em%
%  \begin{center}%
  \let \footnote \thanks
    {\fontsize{18}{20}\selectfont\raggedright  \setlength{\parindent}{0pt} \@title \par}%
}
%\fi
\makeatother
